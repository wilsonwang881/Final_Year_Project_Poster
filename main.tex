\documentclass[12pt, a1paper, landscape, margin=10mm, innermargin=15mm, blockverticalspace=15mm, colspace=15mm, subcolspace=8mm]{tikzposter}
\usepackage[utf8]{inputenc}
 
\title{A high level Schematic editor for simplified HDL Entry}
\author{Lei Wang (Wilson)}
\date{June, 27th, 2019}
\institute{Imperial College London}

\usepackage{blindtext}
\usepackage{comment}
\usepackage{graphicx}
\usepackage{booktabs}

\usepackage{tikz}
\usetikzlibrary{shapes.geometric, arrows, positioning, fit,calc}
\newcommand*\circled[1]{\tikz[baseline=(char.base)]{
            \node[shape=circle,draw,inner sep=1pt] (char) {#1};}}
 
% \usetheme{Default}
 
\begin{document}
 
\maketitle

\begin{columns}
    \column{0.33}
    \block{Abstract}
    {The project aims to produce a piece of software that is used as a high level schematic editor for simplified HDL (Hardware Description Language) Entry by the first-year undergraduate students at Imperial College London studying digital electronics doing synchronous FPGA (Field Programmable Gate Array) digital hardware design. The software is very similar to the Visual2 project in structure of the application stack. The project itself is a combination of the third-year High Level Programming course and the first-and-second-year Digital Electronics course. The implementation involves the use of functional programming and the knowledge of Verilog.}
    
    \block{Feature}
    {
    
\begin{itemize}
    \item The software is a graphical design tool for FPGA hardware design. For creating custom logic blocks, user can either fill in truth tables to generate standalone blocks, or connect different logic blocks together to form a higher level block.
    \item The application is designed to be cross-platform, either for production use or development purposes.
    \item Although the source code is in F\#, interop with JavaScript is made possible by Fable.
\end{itemize}
    }
    
    \block{Electron Framework}
    {
\tikzstyle{block} = [rectangle, minimum width=4cm, minimum height=1cm, text centered, draw={rgb:red,0;green,102;blue,168}, fill={rgb:red,0;green,102;blue,168}, text=white]

\tikzstyle{arrow} = [thick,->,>=stealth]

\begin{tikzpicture}[node distance=3cm]

\node (nodejs) [block] {Electron};

\node (webcontent) [block, right of=nodejs, xshift=6cm] {Web content};

\node (user) [block, right of=webcontent, xshift=6cm] {User};

\node (browser) [block, below of=user, yshift=-1cm] {Electron window};

\node (systemapi) [block, below of=nodejs, yshift=-1cm] {System API};

\draw [arrow] (nodejs) -- (webcontent) node[midway, fill=white] {\circled{1}Deliver};

\draw [arrow] (webcontent) -- (browser) node[midway, fill=white] {\circled{2}Render and Display in};

\draw [arrow] (user) -- (browser) node[midway, fill=white] {\circled{3}React to};

\draw [arrow] (browser) -- ++ (-8, 0) node[midway, fill=white] {\circled{4}Send user response to} -- (nodejs) ; 

\draw [arrow] (nodejs) -- (systemapi) node[midway, fill=white] {\circled{5}Interact with};

\end{tikzpicture}    

    }
    
    \block{Application Stack}
    {
\tikzstyle{block} = [rectangle, minimum width=4cm, minimum height=1cm, text centered, draw={rgb:red,0;green,102;blue,168}, fill={rgb:red,0;green,102;blue,168}, text=white]

\tikzstyle{blockInclude} = [rectangle, minimum width=5cm, minimum height=1cm, draw=black]

\tikzstyle{blockIncludeLarge} = [rectangle, minimum width=5cm, minimum height=1cm, draw=black]

\tikzstyle{arrow} = [thick,->,>=stealth]

\begin{tikzpicture}[node distance=2cm]

\node (electronAPI) [block] {Electron API};

\node (JSLib) [block, right of=electronAPI, xshift=6cm] {JavaScript Library};

\node (fableImport) [block, below of=electronAPI] {Fable.Import};

\node (fableJsInterop) [block, below of=JSLib] {Fable.Core.JsInterop};

\node (fsharpProject) [blockInclude, below of=fableImport] {};

\node [anchor=north west] at (fsharpProject.north west) {F\# Projects};

\node (compiler) [block, right of=fsharpProject, xshift=6cm] {Fable compiler};

\node (webpack) [block, below of=compiler] {Webpack};

\node (webpackConfig) [block, right of=compiler, xshift=5cm] {Webpack configuration};

\node (js) [blockIncludeLarge, below of=webpack] {};

\node [anchor=north west] at (js.north west) {JavaScript Code};

\node (html) [blockInclude, below of=fsharpProject, yshift=-1.5cm] {};

\node [anchor=north west] at (html.north west) {HTML, CSS};

\node (electron) [block, below of=js] {Electron};

\node (browser) [block, right of=electron, xshift=3cm] {Application};

\draw [arrow] (electronAPI) -- (fableImport);

\draw [arrow] (JSLib) -- (fableJsInterop);

\draw [arrow] (fableImport) -- (fsharpProject);

\draw [arrow] (fableJsInterop) -- (fsharpProject.north east);

\draw [arrow] (fsharpProject) -- (compiler);

\draw [arrow] (compiler) -- (webpack);

\draw [arrow] (webpackConfig) |- (webpack);

\draw [arrow] (webpack) -- (js);

\draw [arrow] (html) |- (electron);

\draw [arrow] (js) -- (electron);

\draw [arrow] (electron) -- (browser);

\end{tikzpicture}

    }
 
    \column{0.33}
    \block{Electron Processes and F\# Projects}
    {
\tikzstyle{blockInclude} = [rectangle, minimum width=22cm, minimum height=7cm, draw=black]

\tikzstyle{blockSmall} = [rectangle, minimum width=6cm, minimum height=6cm, draw=black]

\tikzstyle{block} = [rectangle, minimum width=4cm, minimum height=1cm, text centered, draw={rgb:red,0;green,102;blue,168}, fill={rgb:red,0;green,102;blue,168}, text=white]

\begin{tikzpicture}[node distance=4cm]

\node (electronapplication) [blockInclude] {};

\node [anchor=north west] at (electronapplication.north west) {Electron application};

\node [anchor=east] at (electronapplication.east) {More windows...};

\node (page1) [blockSmall, right of=electronapplication, xshift=-7cm] {};

\node [anchor=north west] at (page1.north west) {Window 1};

\node (page2) [blockSmall, right of=page1, xshift=3cm] {};

\node [anchor=north west] at (page2.north west) {Window 2};

\node (main) [block, right of=electronapplication, xshift=-12.5cm, yshift=1.5cm] {Main process};

\node (renderer1) [block, right of=main, xshift=1.5cm] {Renderer process 1};

\node (html1) [block, below of=renderer1, yshift=2.5cm] {HTML file 1};

\node (style1) [block, below of=html1, yshift=2.5cm] {Styling file 1};

\node (renderer2) [block, right of=renderer1, xshift=3cm] {Renderer process 2};

\node (html2) [block, below of=renderer2, yshift=2.5cm] {HTML file 2};

\node (style2) [block, below of=html2, yshift=2.5cm] {Styling file 2};

\end{tikzpicture}    

Each window has one renderer process. The application can have multiple windows hence multiple renderer processes. However, the application can only have one main process. 

Each process corresponds to one F\# project.
    }
    
    \block{Application Windows}
    {
The main window has the following structure:

\begin{tikzpicture}[node distance=1.5cm]

\node (window) [rectangle, minimum width=16cm, minimum height=9cm, draw=black] {};

\node [anchor=west] at (-8,4.25) {Title bar};

\node [anchor=west] at (-8,3.75) {Menu bar};

\node [anchor=west] at (-8,3.15) {Tabs};

\node [anchor=west] at (4,2.5) {Information pane};

\node [anchor=west] at (4,-1.3) {Block pane};

\node [anchor=west] at (-2.5,-1) {Canvas};

\draw (-8,4) -- (8,4);

\draw (-8,3.5) -- (8,3.5);

\draw (-8,2.8) -- (8,2.8);

\draw (4,2.8) -- (4,-4.5);

\draw (4,-1) -- (8,-1);

\end{tikzpicture}

\tikzstyle{arrow} = [thick,->,>=stealth]

The window for creating basic logic blocks has the following structure:

\begin{tikzpicture}[node distance=1.5cm][Window to add basic logic blocks]

\node (window) [rectangle, minimum width=14cm, minimum height=15cm, draw=black] {};

\node [anchor=west] at (-7,7.2) {Create new logic blocks};

\node [anchor=west] at (-7,6) {Type:};

\node (type) [rectangle, minimum width=4cm, minimum height=0.8cm, draw=black] at (-3.5,6) {};

\node (enterName) [anchor=west] at (7.5,6) {\circled{1}Enter block name};

\draw [arrow] (enterName) -- (type);

\node [anchor=west] at (-7,4.6) {Number of inputs:};

\node (enterInput) [anchor=west] at (7.5,4.6) {\circled{2}Enter number of inputs};

\node (inputNumber) [rectangle, minimum width=4cm, minimum height=0.8cm, draw=black] at (-0.8,4.6) {};

\draw [arrow] (enterInput) -- (inputNumber);

\node [anchor=west] at (-7,3.2) {Number of outputs:};

\node (outputNumber) [rectangle, minimum width=4cm, minimum height=0.8cm, draw=black] at (-0.6,3.2) {};

\node (enterOutput) [anchor=west] at (7.5,3.2) {\circled{3}Enter number of outputs};

\draw [arrow] (enterOutput) -- (outputNumber);

\draw (-7,6.9) -- (7,6.9);

\draw (-7,2.5) -- (7,2.5);

\node [anchor=west] at (-7,1.5) {in0};

\node (in0) [rectangle, minimum width=4cm, minimum height=0.8cm, draw=black] at (-4,1.5) {};

\node (namePorts) [anchor=west] at (7.5,1.5) {\circled{4}Name each port.*};

\draw [arrow] (namePorts) -- (in0);

\node [anchor=west] at (-7,0.1) {...};

\draw (-7, -0.5) -- (7,-0.5);

\node [anchor=west] at (-7, -1) {Truth table};

\node (truthTable) [rectangle, minimum width=2cm, minimum height=0.5cm, draw=black] at (-1.5,-1) {Configure truth table};

\node (truthTableButton) [anchor=west] at (7.5,-1) {\circled{5}Click to expand truth table. Fill in.};

\draw [arrow] (truthTableButton) -- (truthTable);

\draw (-7, -3.5) -- (7,-3.5);

\node [anchor=west] at (-7, -4) {Block diagram};

\node (blockView) [rectangle, minimum width=2cm, minimum height=0.5cm, draw=black] at (-1,-4) {Generate block view};

\node (truthTableButton) [anchor=west] at (7.5,-4) {\circled{6}(Optional) Click to view block shape.};

\draw [arrow] (truthTableButton) -- (blockView);

\node (OK) [rectangle, minimum width=1cm, minimum height=0.5cm, draw=black] at (3.5,-7) {OK};

\node (Cancel) [rectangle, minimum width=1cm, minimum height=0.5cm, draw=black] at (5.5,-7) {Cancel};

\node (OKClick) [anchor=west] at (7.5,-6) {\circled{7}Click to submit data.};

\draw [arrow] (OKClick) -| (OK);

\end{tikzpicture}

There is also an About window to display information on author and versions of libraries. It is not put in a dedicated F\# project but instead implemented using the \textit{dialog.showMessageBox} API from Electron.
    }

    \column{0.33}
    \block{Application Functions}
    {
\tikzstyle{block} = [rectangle, minimum width=4cm, minimum height=1cm, text centered, draw={rgb:red,0;green,102;blue,168}, fill={rgb:red,0;green,102;blue,168}, text=white]

\tikzstyle{arrow} = [thick,->,>=stealth]

\centering
\begin{tikzpicture}[node distance=3cm]

\node (truthtable) [block] {Truth table};

\node (blocks) [block, below of=truthtable, yshift=1cm] {Buttom-level blocks};

\node (otherblocks) [block, below of=blocks, yshift=1cm] {Other high-level blocks};

\node (highlevelblocks) [block, right of=blocks, xshift=3cm] {High-level blocks};

\node (Verilog) [block, right of=highlevelblocks, xshift = 3cm, yshift=-1.5cm] {Verilog};

\node (highleveldesign) [block, above of=Verilog] {Higher-level design};

\draw [arrow] (truthtable) -- (highlevelblocks);

\draw [arrow] (blocks) -- (highlevelblocks);

\draw [arrow] (otherblocks) -- (highlevelblocks);

\draw [arrow] (highlevelblocks) -- (Verilog);

\draw [arrow] (highlevelblocks) -- (highleveldesign);

\end{tikzpicture}
    }
    
    \block{IPC}
    {
\tikzstyle{block} = [rectangle, minimum width=4cm, minimum height=1cm, text centered, draw={rgb:red,0;green,102;blue,168}, fill={rgb:red,0;green,102;blue,168}, text=white]

\tikzstyle{arrow} = [thick,->,>=stealth]

\begin{tikzpicture}[node distance=3cm]

\node (renderer) [block] {Renderer process};

\node (main) [block, right of=nodejs, xshift=6cm] {Main process};

\draw [arrow] (renderer) -- ++ (0, 1) -- ++ (7, 0) node[midway, fill=white] {IPCRenderer} -- (main) ; 

\draw [arrow] (main) -- ++ (0, -1) -- ++ (-7, 0) node[midway, fill=white] {IPCMain} -- (renderer);

\end{tikzpicture}    

\begin{tikzpicture}[node distance=3cm]

\node (renderer) [block] {Renderer process};

\node (main) [block, right of=renderer, xshift=6cm] {Main process};

\node (systemAPI) [block, right of=main, xshift=6cm] {System API};

\draw [arrow] (renderer) -- (main) node[midway, fill=white] {\circled{1}\textit{remote} call};

\draw [arrow] (main) -- ++ (0, 1) -- ++ (7, 0) node[midway, fill=white] {\circled{2}API call} -- (systemAPI);

\draw [arrow] (systemAPI) -- ++ (0, -1) -- ++ (-7, 0) node[midway, fill=white] {\circled{3}Response} -- (main);

\end{tikzpicture}

\begin{tikzpicture}[node distance=3cm]

\node (renderer1) [block] {Renderer process 1};

\node (main) [block, right of=renderer, xshift=6cm] {Main process};

\node (renderer2) [block, right of=main, xshift=6cm] {Renderer process 2};

\draw [arrow] (renderer1) -- (main) node[midway, fill=white] {\circled{1}\textit{IPCRenderer}};

\draw [arrow] (main) -- (renderer2) node[midway, fill=white] {\circled{2}\textit{IPCMain}};

\end{tikzpicture}

IPC is the abbreviation for Inter-Process Communication. It is used when the communication between processes is needed. The application uses IPC to pass information between windows using the main process as a relay.

    }
    
    \block{Project Outcome}
    {
    \begin{itemize}
        \item F\# bindings for using \textit{JointJS}.
        \item An application that generates Verilog for synchronous digital hardware design from a graphical design and truth tables.
        \item A proof that \textit{F\# + Electron} is a feasible way to implement desktop application. The desktop application combines the JavaScript ecosystem and the static typing feature of function programming together with the help of Fable.
    \end{itemize}
    }
    
\end{columns}
 
\end{document}
